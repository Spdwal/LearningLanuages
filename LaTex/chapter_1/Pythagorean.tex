%-*- coding: UTF-8 -*-
% gougu.tex
% 勾股定理

\documentclass[UTF8]{ctexart}
\usepackage{graphicx}
\usepackage{amsmath}
\newtheorem{thm}{定理}
\title{杂谈勾股定理}
\author{张三}
\date{\today}

\bibliographystyle{plain}

\begin{document}
	\maketitle
	\begin{abstract}
		这是一篇关于勾股定理的小短文。
	\end{abstract}
	\tableofcontents
	\section{勾股定理在古代}
	西方称勾股定理为毕达哥拉斯定理,将勾股定理的发现归功于公元前6世纪的
毕达哥拉斯学派\cite{Kline}。该学派得到了一个法则,可以求出可排成直角三角形三边的三 元数组。毕达哥拉斯学派没有书面著作,该定理的严格表述和证明则见于欧几里 德《几何原本》的命题47:“直角三角形斜边上的正方形等于两直角边上的两
个正方形之和。”证明是用面积做的。 我国《周髀算经》载商高(约公元前12世纪)答周公问:
	\begin{quote}
	\zihao{-5}\kaishu
		勾广三,股修四,径隅五。
	\end{quote}
	又载陈子(约公元前 7--6 世纪)答荣方问:
	\begin{quote}
	\zihao{-5}\kaishu
		若求邪至日者,以日下为勾,日高为股,勾股各自乘,并而开方除之,得邪至日
	\end{quote}
	都较古希腊更早。后者已经明确道出勾股定理的一般形式。图1是我国古代对勾股定理的一种证明\cite{quanjing}。
	\footnote{欧几里得,约公元前330--275年。}


	%\begin{figure}[ht]
	%	\centering
	%	\includegraphics[scale=0.6]{xiantu.pdf}
	%	\caption{宋赵爽在《周髀算经》注中作的弦图(仿制),该图给出了勾股定理的一个极具对称美的证明。}	
	%	\label{fig:xiantu}	
	%\end{figure}
	

	\section{勾股定理的近代形式}
	
	
	\begin{thm}[勾股定理]
	直角三角形斜边的平方等于两 腰的平方和。
可以用符号语言表述为:设直角三角形 ABC, 其中$\angle ACB = \pi / 2$,则有	
	\end{thm}

	
	\begin{equation}\label{eq:gougu}
		AB^2 = BC^2 + AC^2
	\end{equation}
	
	满足式\eqref{eq:gougu}的整数称为\emph{勾股数}。第 1 节所说毕 达哥拉斯学派得到的三元数组就是勾股数。下表列 出一些较小的勾股数:
		
	\begin{table}[h].            %必须用 h
		\begin{tabular}{|rrr|}
		\hline
		直角边 $a$ & 直角边 $b$ & 斜边 $c$\\
		\hline 
		3 & 4 & 5 \\
		5 & 12 & 13 \\
		\hline	
		\end{tabular}%
		\qquad
		($a^2 + b^2 = c^2$)
	\end{table}
	
	
	
	\nocite{Shiya}
	\bibliography{math}
	


\end{document}